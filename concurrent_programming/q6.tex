\newtheorem{theorem}{Claim}

The code for the sorter, and the testing rig, follow:

\lstinputlisting[language=scala]{Q6.scala}

\subsection{Finding the number of sequentially ordered messages}

\begin{theorem}
    If we sort $n$ integers, then the largest set of messages through \texttt{inner\_channels} that must be sequentially ordered is of size $n$.
\end{theorem}
\begin{proof}
Let $M(i, j)$ denote the $i$'th message passed through \texttt{inner\_channels(j)}. \\
First, there exist $n$ messages that must happen in a particular order; more precisely: $M(1, 0) \prec M(2, 0) \prec M(3, 0) \prec ... \prec M(n, 0)$, where $a \prec b$ asserts that $a$ must happen strictly before $b$. \\
Second, there exists no set of more than $n$ messages that must be ordered sequentially, as the following ordering of messages is possible:
\begin{itemize}
\item First $M(1, 0)$.
\item Then $M(2, 0), M(1, 1)$ simultaneously.
\item Then $M(3, 0), M(2, 1), M(1, 2)$ simultaneously.
\item ...
\item Then $M(n, 0), M(n-1, 1), ..., M(1, n-1)$ simultaneously.
\end{itemize}
And the longest sequentially ordered chain of messages in this ordering has length $n$. \\
So, overall, the largest set of sequentially ordered messages is of size $n$.
\end{proof}

If we take into account the messages through \texttt{out}, this gives us $n+1$ sequentially ordered messages overall.
