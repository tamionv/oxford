\section{Problem 2}
\subsection{a}

To do this, simply use the following sequence of instructions:\\
\ASMEnv{
    \ASMRow{Instruction}{Stack after instruction}
    \ASMRow{CONST 1}{1}
    \ASMRow{LDGW \_x}{1, x}
    \ASMRow{PLUS}{1+x}
    \ASMRow{CONST 1}{1+x, 1}
    \ASMRow{SWAP}{1, 1+x}
    \ASMRow{DIV}{1/(1+x)}
}

\subsection{b}
\Proof{Supposing that:
\begin{center}
$cost(x) = \text{optimal depth for some expression } x$
\end{center}
Then: 
\begin{center}
$cost(Binop(w, e_1, e_2)) = min \{ max \{ cost(e1), cost(e2)+1\}, max\{ cost(e1)+1, cost(e2) \} \}$
\end{center}
}{
    Note that in order to evaluate $Binop(w, e_1, e_2)$, we must first evaluate $e_1$ and $e_2$ in some order (we can do either first because of the \texttt{SWAP} instructions). Note that if we evaluate, say, $e_1$ first, then evaluating $e_2$ will lead to a stack height one larger for $e_2$ than it would otherwise (due to the result of evaluating $e_1$). So, evaluating $e_1$ first leads to a stack depth of $max \{ cost(e1), cost(e2) + 1 \}$. Symmetrically, evaluating $e_2$ first leads to a stack of depth $max \{ cost(e1)+1, cost(e2) \}$. Taking the smaller of these two leads us to the claim. Note that I ignore the possibility of using the information in one expression to calculate the other. }
Also, this is an \texttt{Ocaml} definition of the function above:

\begin{lstlisting}[language=Ml]
let rec cost = function
      Variable str      -> 1
    | Constant v        -> 1
    | Monop (w, y)      -> cost y
    | Binop (w, e1, e2) -> min
        (max (cost e1) (1 + cost e2))
        (max (1 + cost e1) (cost e2))


\end{lstlisting}
\leavevmode
\\

\newpage
\Proof{$operands(e) < 2^N \Rightarrow cost(e) \leq N$, where $operands(e) = \text{the number of operands in }e$}{
    I prove this claim by induction on $|e|$, for all $N$, where $|e| = \text{the number of operands and operators in }e$: \\
    \Induction{
        Assume $|e| = 1$. Assume also that $2^N > operands(e)$. Then $2^N > operands(e) \geq |e| = 1$, so $N > 0$. Also, since $|e| = 1$, $e$ is either \texttt{Variable} or a \texttt{Constant}, so $cost(e) = 1 = 2^0 < 2^N$.
    }{
        Supposing that the claim is true for all expressions $f$ with $|f| < n$ (where $n > 1$), then, for some expression $e$ with $|e| = n$, we have several cases:
        \begin{itemize}
            \item if $e$ is a \texttt{Variable} or \texttt{Constant}, $operands(e) < 2^N \Rightarrow cost(e) \leq N$ is vacuously true, since no \texttt{Variable} or \texttt{Constant} has more than $1$ operands or operators, yet $|e| = n > 1$.
            \item if $e = \texttt{Monop(}w, f\texttt{)}$, then by definition, $operands(e) = operands(f)$ and $cost(e) = cost(f)$. Since, by the inductive hypothesis, $operands(f) < 2^N \Rightarrow cost(f) \leq N$, then using the previous identities we get that: $operands(e) < 2^N \Rightarrow cost(e) \leq N$.
            \item if $e = \texttt{Binop(}w, f, g\texttt{)}$. Assume $operands(e) < 2^N$ for some $N$. Let $M$ and $K$ be the unique integers that satisfy $2^{M-1} \leq operands(f) < 2^M$ and $2^{K-1} \leq operands(g) < 2^K$. W.l.o.g. let $M \leq K$. Assume for contradiction that $N \leq M$. Then
                \begin{align*} operands(e) &= 1 + operands(f) + operands(g) && \text{(by definition of $operands$)}\\
                    &\geq 1 + 2^{M-1} + 2^{K-1} && \text{(definition $M, K$)}\\
                                     &\geq 1 + 2^{N-1} + 2^{N-1} && \text{(assumption and transitivity)}\\
                                     &> 2^N \end{align*}
                a contradiction. So $N > M$. Assume for contradiction that $N < K$. Then
                \begin{align*} operands(e) &= 1 + operands(f) + operands(g) && \text{(as before)}\\
                    &\geq 1 + 2^{M-1} + 2^{K-1} && \text{(as before)}\\
                                      &\geq 1 + 0 + 2^{(N+1)-1} && (N < K \Rightarrow K \geq N+1))\\
                                      &> 2^N \end{align*}
                a contradiction.  So $N \geq K$. Use the inductive hypothesis on $f, M$ and $g, K$ to get that $cost(f) \leq M$ and $cost(g) \leq K$. Now
                \begin{align*} cost(e) &= min \{ max \{ cost(f), 1 + cost(g) \}, max \{ 1 + cost(f), 1 + cost(g) \} \}\\
                                       &\leq min \{ max \{ M, 1 + K \}, max \{ 1+M, K \} \}\\
                                       &\leq min \{ max \{ N-1, N+1 \}, \{ 1 + N-1, N \} \}\\
                &\leq N \end{align*}
        \end{itemize}
    }
    So, as the base case \textit{BC} holds, and the inductive step \textit{IS} works in all cases, the claim is true
}

\newpage
\subsection{c}

\begin{lstlisting}[language=ML]
let rec gen_expr =
    function
      Constant x ->
        CONST x
    | Variable x ->
        SEQ [LINE x.x_line; LDGW x.x_lab]
    | Monop (w, e1) ->
        SEQ [gen_expr e1; MONOP w]
    | Binop (w, e1, e2) ->
        if cost e1 >= cost e2 then
            SEQ [gen_expr e1; gen_expr e2; BINOP w]
        else
            SEQ [gen_expr e2; gen_expr e1; SWAP; BINOP w]
\end{lstlisting}
