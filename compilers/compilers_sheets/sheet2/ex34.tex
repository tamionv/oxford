\section{Problem 3.4}
It is impossible to be able to do this with perfect accuracy at compile-time, because the variables that are initialised at any time might depend on information known only at run-time (such as user-input). I would err on the side of being more restrictive rather than less restrictive -- forcing variables to always have some though-out assigned value before use makes programmers think about these edge cases, which eliminates a whole class of bugs. Thus I would enforce a rule that a variable must be assigned before use no matter what the execution path through the code. The most difficult control structure to deal with is \texttt{exit}, as only these break normal control flow. A function that checks this property follows:

\begin{lstlisting}[language=Ml]

let intersect2 xs ys = filter (fun x -> mem x xs) ys
let intersect (xs :: xss) = fold_right intersect2 xss xs 

let has_exit = function
    Seq (x :: xs) -> has_exit x || has_exit (Seq xs)
    | Exit -> true
    | IfStmt(e, ifpt, elsept) -> has_exit ifpt || has_exit elsept
    | WhileStmt(test, body) -> has_exit body
    | RepeatStmt(body, test) -> has_exit body
    | LoopStmt body -> false
    | CaseStmt (switch, cases, default) ->
        has_exit default || exists has_exit (map snd cases)
    | _ -> false

let will_exit = function
    Seq (x :: xs) -> will_exit x || will_exit (Seq xs)
    | Exit -> true
    | IfStmt(e, ifpt, elsept) -> will_exit ifpt && will_exit elsept
    | WhileStmt _ -> false
    | RepeatStmt (body, test) -> will_exit body
    | LoopStmt body -> false
    | CaseStmt (switch, cases, default) ->
        will_exit default && for_all will_exit (map snd cases)
    | _ -> false

(* f takes an expression and returns the variables it uses
 * g takes a statement and returns a tuple: the variables it
 * needs to be initialised prior, and the variables it
 * certainly initialises *)
let test_stmt = 
    let rec f = function
        Constant _ -> []
        | Variable x -> [x.x_lab]
        | Monop (_, e) -> f e
        | Binop (_, e1, e2) -> f e1 @ f e2 in
    let rec g = function
        Skip -> ([], [])
        | Seq (x :: xs) ->
            (* A sequence needs everything that the first part needs,
             * together with everything the second part needs that is
             * not defined in the first part.
             * A sequence defines everything that the first or second
             * part defines
             * SPECIAL CASE: if the first thing will exit, then ignore
             * the rest. If the first thing might exit, then we
             * must take into account the entire Seq for needs,
             * and only the first part for things it defines *)
            let (usedl, defsl) = g x
            and (usedr, defsr) = g (Seq xs)
            and used = usedl @ (filter (fun x -> not (mem x defsl)) usedr)
            and defs = defsl @ defsr
            in if will_exit x then (usedl, defsl)
            else if has_exit x then (used, defsl)
            else (used, defs)
        | Seq [] -> ([], [])
        | Assign (v, e) -> (f e, [v.x_lab])
        | Print e -> (f e, [])
        | Newline -> ([], [])
        | IfStmt (test, thenpt, elsept) ->
            (* An if statement needs everything any of its clauses
             * needs, and defines only what both of its clauses
             * defines *)
            let used1 = f test
            and (used2, defs2) = g thenpt
            and (used3, defs3) = g elsept
            in (used1 @ used2 @ used3
               , intersect2 defs2 def3)
        | WhileStmt (test, body) ->
            (* a while statement needs everything its test or
             * body might need, and defines nothing (as the body
             * might not even be run *)
            let used1 = f test
            and (used2, _) = g body
            in (used1 @ used2, [])
        | RepeatStmt (body, test) ->
            (* a repeat statement is like a while, just that
             * it defines everything its body defines, since
             * the body is guaranteed to be run *)
            let used1 = f test
            and (used2, defs) = g body
            in (used1 @ used2, defs)
        | ExitStmt -> ([], [])
        | LoopStmt body ->
            (* A loop statement needs everything its body needs,
             * and defines what it's body defines *)
             g body
        | CaseStmt (switch, cases, default) ->
            (* A case statement needs everything any of its
             * body or switch needs, and defines whatever
             * all of the cases defines for sure *)
            let used1 = f switch
            and used2 = concat (map fst (map g (map snd cases)))
            and defs2 = map snd (map g (map snd cases))
            and (used3, defs3) = g default
            in (used1 @ used2 @ used3, intersect (defs3 :: defs2))
    in function x -> length (fst (g x)) == 0

\end{lstlisting}
