\subsection{}
The trees are found in diagram 12.

\subsection{}
The trees are found in diagram 13-14. Diagram 14 also tiles the trees.

\subsection{}
The main assumption is that the assignments a[i] := a[j], a[j] := t do not change the values of i and j. This is justified, as integer parameters are copied onto the stack, and thus should not alias any position in an array, or the local variable t. A conservative alias analyser might miss this observation.

\subsection{}
\begin{lstlisting}
!Put _a into r0
ldr r0, =_a

!Put i = [fp, #40] into r1
ldr r1, [fp, #40]

!Put _a+4i into r1
add r1, r0, r1, LSL 2

!Put j = [fp, #44] into r2
ldr r2, [fp, #44]

!Put _a+4j into r2
add r2, r0, r2, LSL 2

!Put a[i] into t=r4
ldr r4, [r1]

!Put a[j] into r3
ldr r3, [r2]

!Store a[j] in a[i]
str r3, [r1]

!Store r4=t in a[j]
str r4, [r2]
\end{lstlisting}

\subsection{}
I believe this code to be optimal, as no common subexpressions are not shared, and since all trees in our sequence have been converted into instructions in an optimal way.
