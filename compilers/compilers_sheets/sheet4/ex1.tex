The rules mentioned (on lines 1, 4, 6, 9, 17, 22, 36–40, 42–44 and 49), are:

\enumerate{
\item $reg  \rightarrow \langle Const\text{ } k \rangle$ (when fits move k)
\item $reg  \rightarrow \langle Local\text{ } n \rangle$ (when fits add n)
\item $reg  \rightarrow \langle Global\text{ } x \rangle$
\item $reg  \rightarrow \langle Loadw, addr \rangle$
\item $reg  \rightarrow \langle Binop\text{ } PlusA, reg1 , rand \rangle$
\item $reg  \rightarrow \langle Binop\text{ } Lsl, reg1, rand \rangle$
\item $rand \rightarrow \langle Const\text{ } k \rangle$ (when fits immed k)
\item $rand \rightarrow \langle Binop\text{ } Lsl, reg, \langle Const\text{ } n \rangle \rangle$ (when $n < 32$)
\item $rand \rightarrow \langle Binop\text{ } Lsl, reg1, reg2 \rangle$
\item $rand \rightarrow reg$
\item $addr \rightarrow \langle Local\text{ } n \rangle$ (when fits offset n)
\item $addr \rightarrow \langle Binop\text{ } PlusA, reg1, reg2 \rangle$
\item $addr \rightarrow \langle Binop\text{ } PlusA, reg1, \langle Binop\text{ } Lsl, reg2, \langle Const\text{ } n \rangle \rangle \rangle$ (when $n < 32$)
\item $addr \rightarrow reg$
\item $stmt \rightarrow \langle Storew, reg, addr \rangle$
}

\leavevmode
\\

To systematically calculate the number of ways in which we can tile the tree with these rules, we use dynamic programming:
\begin{itemize}
\item For each node $i$, and each symbol $S \in \{ stmt, addr, rand, reg \}$ we calculate $d_i^S$, the number of ways in which we can tile the subtree whose root is at $i$, such that the subtree can be generated from the symbol $S$.
\item To find $d_i^S$, assuming that $d_j^{S'}$ has been found for all nodes $j \ne i$ in the subtree of $i$ and all symbols $S'$, we use the following formula: \\
    $d_i^S = \sum\limits_{p\text{ is a pattern for } S \text { that matches subtree at } i} (\prod\limits_{X \text{is non-terminal in p}} d_{\text{node where } X \text{ falls}}^X)$
\end{itemize}
\leavevmode
\par

The answer is in $d_{root}^{stmt}$. \\
Numbering the nodes as in the diagram 1, the following table synthesizes the information found by the dynamic programming algorithm:

\begin{tabular}{|c|c|c|c|}

\hline
Node $i$ & Symbol $S$ & Rules & $d_i^S$ \\
\hline
1 & reg & & 0 \\
\hline
1 & rand & 10 & 0 \\
\hline
1 & addr & 14 & 0 \\
\hline
1 & stmt & 15 & \underline{28} \\
\hline

2 & reg & 4 & 14\\
\hline
2 & rand & 10 & 14\\
\hline
2 & addr & 14 & 14\\
\hline
2 & stmt & & 0 \\
\hline

3 & reg & 2 & 1\\
\hline
3 & rand & 10 & 1\\
\hline
3 & addr & 11, 14 & 2\\
\hline
3 & stmt & & 0 \\
\hline

4 & reg & 5 & 8\\
\hline
4 & rand & 10 & 8\\
\hline
4 & addr & 12, 13, 14& 14 \\
\hline
4 & stmt & & 0 \\

\hline
5 & reg & 3 & 1\\
\hline
5 & rand & 10 & 1\\
\hline
5 & addr & 14 & 1\\
\hline
5 & stmt & & 0 \\
\hline

6 & reg & 6 & 4\\
\hline
6 & rand & 8, 9, 10& 8\\
\hline
6 & addr & 14 & 4 \\
\hline
6 & stmt & & 0 \\
\hline

7 & reg & 4 & 2 \\
\hline
7 & rand & 10 & 2 \\
\hline
7 & addr & 14 & 2 \\
\hline
7 & stmt & & 0 \\
\hline

8 & reg & 1 & 1 \\
\hline
8 & rand & 7, 10 & 2 \\
\hline
8 & addr & 14 & 1\\
\hline
8 & stmt & & 0 \\
\hline

9 & reg & 2 & 1 \\
\hline
9 & rand & 10 & 1 \\
\hline
9 & addr & 11, 14& 2\\
\hline
9 & stmt & & 0 \\
\hline
\end{tabular}
\\
\\
So the answer is that there are 28 different tilings. \\

By inspecting these, I've found that the shortest code is:

\begin{lstlisting}{language=arm}
ldr r0, =_a
ldr r1, [fp, #40]
ldr r2, [r0, r1, LSL #2]
str r2, [fp, #-8]
\end{lstlisting}

And that the worst code is the one that uses the least specific rule at each possible choice (i.e. splits each node into its own tile), and that stores all temporary values in registers:

\begin{lstlisting}{language=arm}

ldr r0, =_a
add r1, fp, #40
ldr r2, [r1]
mov r3, #2
lsl r4, r2, r3
add r5, r0, r4
ldr r6, [r5]
add r7, fp, #-8
str r6, [r7]
\end{lstlisting}
