\documentclass[a4paper,11pt]{scrartcl}
\usepackage{listings}
\usepackage{color}

\usepackage{amsmath,amssymb,amsthm} 

\begin{document}

\section{}
\subsection{}
Removing preimage resistance has no effect. If we are trying to forge a signature for a message, we already have it, and we don't need to deduce it from the hash.
\subsection{}
Removing 2nd-preimage resistance leads to an EF under CMA, as follows: as our hash fails 2nd-preimage resistance, suppose that given some message $x$, we can easily find a message $y$ such that $h(x) = h(y)$, where $h$ is the hash. Then, if we can convince the signer to sign the message $x$, it's signature will coincide will the signature for $y$. This provides the forgery. This attack is not adaptive.
\subsection{}
Removing collision resistance leads to an EF under CMA. This is the same attack as the previous one, just on the pair $x, y$ that invalidates collision resistance.

\section{}
\subsection{}
\underline{Proof:} If the signature is valid, then: 
    $r \equiv g^k (\text{mod} p) ^ s \equiv (m - xr) k ^ {-1} (\text{mod} p - 1)$
Now:
\begin{itemize}
\item as $1 < g < p$, and as $\{1, 2, ..., p-1\}$ is closed under multiplication modulo $p$ (as $p$ is prime), inductively, $r \equiv g^k \not\equiv 0$.

\end{document}


